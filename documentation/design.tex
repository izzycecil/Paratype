\documentclass[10pt]{article}
\usepackage[margin=1in]{geometry}

% no paragraph indentation
\setlength{\parindent}{0.0in}
\setlength{\parskip}{6pt}

% header stuff
\usepackage{fancyhdr} % fancy headers
\pagestyle{fancy}
\fancyhead{}
\fancyfoot{}

\fancyhead[R]{Design Documentation}
\fancyhead[L]{Paratype}
\fancyfoot[L]{WIP --- Cecil, Turrubiates, Koch}
\fancyfoot[R]{\thepage}
\renewcommand{\headrulewidth}{0.4pt}
\renewcommand{\footrulewidth}{0.4pt}

% captions with bold labels
\usepackage[format=plain,font=small,labelfont=bf]{caption}

% noitemsep option for enumerate and itemize
\usepackage{enumitem}

% mathbb
\usepackage{amsmath,amssymb}

% hyperlinks in table of contents
\usepackage[hidelinks]{hyperref}

% typesetting algorithms
\usepackage{algorithmicx}
\usepackage[noend]{algpseudocode}
\usepackage[section]{algorithm}
\usepackage[nottoc,numbib]{tocbibind}


% Grammar
\usepackage{syntax, etoolbox}
\AtBeginEnvironment{grammar}{\small}
\setlength{\grammarparsep}{8pt plus 1pt minus 1pt}
\setlength{\grammarindent}{12em}

% strike through text
\usepackage{soul}

% typesetting 'real code'
\usepackage{listings,color,xcolor,MnSymbol}
\definecolor{dkgreen}{rgb}{0,0.6,0}
\definecolor{dkblue}{rgb}{0,0,0.6}
\definecolor{cyan}{rgb}{0,0.6,0.6}
\definecolor{maroon}{rgb}{0.5,0,0}

\lstset{
	prebreak=\raisebox{0ex}[0ex][0ex]{\ensuremath{\rhookswarrow}},
	postbreak=\raisebox{0ex}[0ex][0ex]{\ensuremath{\rcurvearrowse\space}},
	breaklines=true,
	numbers=left,
	numberstyle=\scriptsize,
	breakatwhitespace=true,
	frame=single,
	tabsize=4,
	showstringspaces=false,
	aboveskip=0.5em,
	belowskip=0.5em,
	captionpos=b,
	xleftmargin=0.5em,
	xrightmargin=0em,
	keywordstyle=\bfseries\color{dkgreen},
	commentstyle=\itshape\color{purple},
	identifierstyle=\color{maroon},
	stringstyle=\color{blue},
	basicstyle=\small\ttfamily
}

\lstdefinelanguage{Paratype}{
	morekeywords={func,inherits,implements,throws,type,typeclass,constrain,to},
}

\lstdefinelanguage{Gopp}{
	%keywords={=>,=,*,+},
	comment=[l]{\#\#},
	morestring=[b]{'},
%	moredelim=*[s]{(}{)},
%	moredelim=*[s]{<}{>},
}

\begin{document}

\renewcommand{\listofalgorithms}{\begingroup
	\tocfile{List of Algorithms}{loa}
	\endgroup
}
\makeatletter
\let\l@algorithm\l@figure
\makeatother
\pagenumbering{roman}


\title{Paratype --- Design Documentation\\
Work In Progress}
\author{Tyler Cecil, Ben Turrubiates, and Chris Koch}
\date{\today}
\maketitle

\tableofcontents
\listofalgorithms

\pagebreak
\setcounter{page}{0}
\pagenumbering{arabic}

\section{Introduction}
Saying that we evaluate types using an actor model is hardly enough information
to get a good grasp of Paratype. In this document, we attempt to
describe and justify our algorithm, both for our and the reader's benefit.

Paratype is an actor model of \emph{type evaluation}. Type evaluation is our
way of describing two main tasks a compiler of a language using the
Hindley-Milner type system or other related type systems would have: type
checking and producing explicit types.

For example, consider the following two \emph{Haskell} functions.
\begin{lstlisting}[language=Haskell,caption=Haskell example,label=lst:haskell]
map :: (a -> b) -> [a] -> [b]
map _ [] = []
map f (x:xs) = f x : map f xs

positives = (+1) [0,1,..]

badfunction = (+1) "abcdefg"
\end{lstlisting}

The compiler has two basic tasks concerned with types in this example:
determining that \lstinline!badfunction! is a type error and constructing an
integer implementation for \lstinline!map!. In reality, the produced code for
\lstinline!map! is entirely polymorphic, but one could imagine scenarios when
this would be necessary.

With Paratype, we made the following connection: Functions are actors and they
communicate types to other functions.

In many ways this seems like a more natural solution than using Algorithm
W\footnote{Used to solve Hindley-Milner type systems, see
	\url{http://en.wikipedia.org/wiki/Hindley\%E2\%80\%93Milner_type_system\#Algorithm_W}}. 
Eventually we hope to do much more
research than what Paratype covers; this is only a preliminary experiment.
Functions as actors is not enough to understand the relationships between
types.

\section{Architecture}

Paratype is written in Google Go.

In Paratype, every function is an actor. In Go, we represent each actor to be a
\texttt{goroutine}, which is essentially a lightweight thread. The
\textsc{runtime} algorithm presented below is the \texttt{goroutine}. The
functions (\texttt{goroutines}) communicate through a construct in Go called
channels.

\pagebreak
\section{Algorithm}

\subsection{Setup}

Every function $f$ will create its own context object and send a path-context
object $(f, C_f)$ to its child.

The atlas of each $C_f$ must contain objects for $f$ and each of its
children; for example, if $g$ is a child of $f$, then the atlas must contain
$f$ and $f \circ g$ and have the proper type variable relations.

The context object will contain a type map and atlas for itself and the child
function, showing the relations between type variables of the child and itself.

\begin{algorithm}
	\caption{Setting up function $f$}
	\begin{algorithmic}[1]
		\Require{function $f$ and its code}
		\Statex
		\Function{setup}{$f$}
			\State $\mathcal{W}_f \gets \{\}$ \Comment{empty dict --
			typevar-to-typevar map}
			\State $\mathcal{G}_f \gets $ list of child functions (from code)
			\State $\mathcal{L}_f \gets ()$ \Comment{empty list -- list of
			parent contexts}
			\State $\mathcal{E}_f \gets $ set of errors that $f$ throws (from
			code)
			\State $\mathcal{N}_f \gets $ number of parameters of $f$ (from
			code)
			\State $\mathcal{M}_f \gets \{F_0 : \textrm{return type if explicit,
				\texttt{incomplete} otherwise} \}$ \Comment{type map}
				\State $\mathcal{A}_f \gets \{ f : (F_0) \}$ \Comment{atlas}
%			\State $V \gets \{\}$ \Comment{map typevar names in code to new
%			typevar names for algorithm}
			\ForAll{parameters of $f$ in order}
				\State $T \gets $ typevar name from code
				\If{$T$ not in $\mathcal{M}_f$} \Comment{not in type map
					yet}

					\State $\mathcal{M}_f[T] \gets $ explicit type or
						\texttt{incomplete}
					\If{type in $\mathcal{M}_f[T]$ unknown (undeclared)}
						\State \Return \textbf{type error}
					\EndIf
					\State $\mathcal{K}_f[T] \gets $ type classes if
					specified, \texttt{resolved} for explicit type, or
					\texttt{any} if unspecified 
					\If{type classes in $\mathcal{K}_f[T]$ unknown
						(undeclared)}
						\State \Return \textbf{type error}
					\EndIf
				\EndIf
%				\Else \Comment{typevar reused}
%					\State $T \gets V[$ typevar name in code $]$
%				\EndIf
				\State $\mathcal{X}_f[$ variable name $] \gets T$ 
				\State $\Call{append}{\mathcal{A}_f[f], T}$ \Comment{add new
					typevariable to tuple}
			\EndFor

			\State $i \gets 1$ \Comment{$i$ is counter for new typevariable names}
			\ForAll{child function $g$ of $f$ (func composition: outer to inner)}
				\If{$g$ is outermost function call}
					\State $\mathcal{A}_f[f \circ g] \gets (F_0)$
					\Comment{Return type}
				\EndIf
				\ForAll{parameters of $g$ in order}
					\If{parameter is explicit type $t$}
						\State $F_i \gets $ new typevar
						\State $\Call{append}{\mathcal{A}_f[f \circ g], F_i}$ \Comment{add new
						typevariable to tuple}
						\State $\mathcal{M}_f[F_i] \gets t$ 

						\If{type $t$ unknown / undeclared}
							\State \Return \textbf{type error}
						\EndIf

						\State $\mathcal{K}_f[F_i] \gets $ \texttt{resolved}
						\Comment{does not matter anyway}
						\State $i \gets i+1$
					\ElsIf{parameter is function call to $h$}
						\State $F_i \gets $ new typevar
						\State $\Call{append}{\mathcal{A}_f[f \circ g], F_i}$
						\Comment{add new typevariable to tuple}
						\State $\mathcal{A}_f[f \circ h] \gets (F_i)$
						\Comment{return type for $h$}
						\State $\mathcal{M}_f[F_i] \gets $ \texttt{incomplete}
						\State $\mathcal{K}_f[F_i] \gets $ \texttt{any}
						\State $i \gets i+1$

					\ElsIf{parameter is variable name}
						\State $\Call{append}{\mathcal{A}_f[f \circ g],
						\mathcal{X}_f[$ variable name $]}$ \Comment{add new
							typevariable to tuple}
					\ElsIf{parameter is typevar name $T$}
						\State $\Call{append}{\mathcal{A}_f[f \circ g], T}$
					\EndIf
				\EndFor
			\EndFor
		\EndFunction
	\end{algorithmic}
\end{algorithm}

\subsection{Runtime}

In the thread corresponding to function $f$:

\begin{enumerate}[noitemsep]
	\item  Receive a path-context object $(P, C)$ (path, context)

	%\item Add information from $C_f$ to $C$

	\item $\Call{update}{C_f, C}$

	\item Add $f$ (myself) to path $P$, send $(P, C)$ to all children (if
		applicable).

	\item Waiting state.
\end{enumerate}

\begin{algorithm}
	\caption{Runtime for function $f$}
	\begin{algorithmic}[1]
		\Require{function $f$ and its code (i.e. part of the AST that is
		translated into objects by GOPP in the actual algorithm)}
		\Statex
		\Function{runtime}{$f$}
		\State state$_f$ $\gets$ running
		\State \Call{setup}{$f$}
		\State state$_f$ $\gets$ waiting
		\While{$(P, C) \gets
			\Call{readBlockingChannel}{f}$}\Comment{path-context object}
			\State state$_f$ $\gets$ running
			\State \Call{update}{$C_f, C$}
			\State $P \gets P \circ f$ \Comment{add myself to path}
			\ForAll{$g \in \mathcal{G}_f$}
				\State send $(P, C)$ to $g$
			\EndFor
			\State state$_f$ $\gets$ waiting 
		\EndWhile
		\If{every thread is in waiting state and all buffers are empty}
			\State \Call{finish}{$f$}
		\EndIf
		\EndFunction
	\end{algorithmic}
\end{algorithm}

%\begin{algorithm}
%	\caption{Merging}
%\begin{algorithmic}[1]
%	\Require{context $C$ and path $P$}
%	\Statex
%	\Function{merge}{$C_f, C, P$}
%	\ForAll{$D$ in $\mathcal{L}(C)$} \Comment{all direct parent contexts}
%	\If{$D$ is in path $P$} \Comment{is $D$ on path that we received context
	%	from?}
%	\State \Call{update}{$C_f, D$}
%	\State \Call{merge}{$C_f, D, P$}
%	\EndIf
%	\EndFor
%	\EndFunction
%\end{algorithmic}
%\end{algorithm}

\begin{algorithm}
	\caption{Replacing type variable $V$ in $C_g$ with type variable $W$ from
	$C_f$}
	\begin{algorithmic}[1]
		\Statex
		\Function{UpdateTypevar}{$C_g, V, C_f, W$}
			\If{$\mathcal{M}_g[ V ] \neq$ \texttt{incomplete} and
			$\mathcal{M}_f[W] \neq$ \texttt{incomplete} and $\mathcal{M}_g[V]
				\neq \mathcal{M}_f[W]$}
				\State\Return type error!
			\EndIf
	
%			\State $\mathcal{T}(C_g)[W] \gets $ incomplete
%				\Comment{add type variable to type map}
			\State \Comment{add new type variable to type map}
			\If{$\mathcal{M}_g[ V ] \neq $ \texttt{incomplete}}
				\State $\mathcal{M}_g[ W ] \gets \mathcal{M}_g[ V ]$
				\Comment{copy explicit type if it exists}
			\Else
				\State $\mathcal{M}_g[ W ] \gets \mathcal{M}_f[ W ]$
			\EndIf

			\State \Comment{merge type classes}
			\State $\mathcal{K}_g[ W ] \gets \mathcal{K}_g[V] \cap
			\mathcal{K}_f[W]$. \Comment{\texttt{Any} $\cap D = D$}
			\If{$\mathcal{K}_g[ W ] = \emptyset$ or $\mathcal{M}_g[W]
			\not \in \mathcal{K}_g[W]$} 
				\State \Return type error! \Comment{if no type class or typevar
			is not part of typeclass}
			\EndIf

			\State $\mathcal{W}_f[V] \gets W$ \Comment{for future reference}

			\State $V \gets W$ \Comment{update type variable in $C_g$}
		\EndFunction	
	\end{algorithmic}
\end{algorithm}

\begin{algorithm}
	\caption{Updating one context with information from another}
	\begin{algorithmic}[1]
		\Require{source context $C_f$ and to-be-updated context $C_g$ ($C_f$ is
		parent of $C_g$ in context tree)}
		\Statex
		\Function{Update}{$C_f, C_g$}
		\If{$g$ is direct parent of $f$} \Comment{match type variables of
		function call to function definition}
		\State $V \gets \mathcal{A}_g[g \circ f]$ 
		\State $W \gets \mathcal{A}_f[f]$
		\For{$i = 0$ to $\Call{length}{V}-1$}
			\If{$V[i] \neq W[i]$}
				\State \Call{UpdateTypevar}{$C_g, V[i], C_f, W[i]$}
			\EndIf
		\EndFor
		\EndIf
		\State	
		\State \Comment{If $g$'s implementation
		contains variables that $f$ has replaced before, replace them}
		\ForAll{$v \in \mathcal{A}_g[g]$} 
			\If{$v \in \mathcal{W}_f$}
			\State \Call{UpdateTypevar}{$C_g, v, C_f, \mathcal{W}_f[v]$}
			\EndIf
		\EndFor
		\State $\mathcal{E}_g \gets \mathcal{E}_g \cup \mathcal{E}_f$
		\Comment{merge error types}
		\EndFunction
	\end{algorithmic}
\end{algorithm}

\subsection{Finish}

The algorithm will be considered finished when all threads are in a waiting
state and when all channel buffers are empty.

At this point, each function will try to resolve any type relations due to
function composition or multiple function calls (currently not supported in
grammar, but semantically the same as function composition in our case since we
are only interested in type variables).

When that is done, every function will walk up the call graph to find explicit
types for its type variables.

\subsection{Type Error Detection}

Type errors are detected when:
\begin{enumerate}
	\item The algorithm tries to merge two incompatible explicit types
	\item The algorithm tries to merge two incompatible sets of type classes
		(given two sets of type class constraints on the same type variable,
		their intersection is empty) 
	\item An explicit type seen in the code is undeclared
	\item An explicit type class seen in the code is undeclared
	\item \ldots
\end{enumerate}

\subsection{Current Problems / To Do}

All of these seem resolvable with the current algorithm.

\begin{itemize}
	\item Sending path-context object to all children -- there may be
		contention for editing the same typevariables?

		No -- because of locks. A child will have to lock down all of $C_f$
		while it edits it, makes its type variable changes, etc. When child 2
		overrides a type variable of child 1 later, it will be adding it to its
		type variable map. 
		
		There is a record of what was changed -- but how do
		we access it nicely? maybe $C_f$ should have that in its type variable
		map for ease of resolution at the finish step?

	\item \st{To add to algorithm: When merging type variables, the type
			class(es) that are associated with them must merge, too. 

			Context should have map of type variables to type classes?} Dealt
			with.

	\item To add to algorithm: When and how do we detect type errors?

		\begin{enumerate}
			\item \st{Type class errors whenever an explicit type is specified}
			\item \st{Type class merge error}
			\item Explicit types that are not the same will not merge -- can
				there be compatible explicit types?
			\item When would type variables not merge?
			\item \ldots?
		\end{enumerate}

	\item \st{To add to algorithm: Error types currently not incorporated
		(should be easy, will require another type of communication though)}

	\item To add to algorithm: Detection of cycles in call graph (recursion,
		mutual recursion, etc) and their type convergence (i.e. all types have
		been enumerated)

		Will converge because there are a finite set of types and thus,
		function calls are a functional iteration on a finite sequence (the
		types). This MUST be ultimately periodic for cycles, and we can stop
		after a cycle is found.

		Just has to be written into algorithm.
\end{itemize}

\pagebreak
\section{Terms and Definitions}

\subsection{General Terminology}
	\paragraph{Function / Context} 
	A function is an element of the set of functions. Functions are equipped
	with parameters and their types, return type, error types, and composed
	function calls.  Functions will be denoted by the lowercase roman letters
	$f$, $g$, and $h$.  Multiple implementations (realizations with different
	explicit types) will be denoted with differentiation notation (e.g.
	Lagrange's notation: $f'$ or $h^{(4)}$ or $h^{iv}$).  The set of functions
	will be denoted $\mathcal{F}$.

	Every function has a \emph{context} which contains information about the
	implementations (explicit types) of itself. A context for a function $f$ is
	a tuple named $C_f = (\mathcal{M}, \mathcal{A}, \mathcal{W}, \mathcal{G},
	\mathcal{X}, \mathcal{L}, \mathcal{K}, \mathcal{E}, \mathcal{N})$, where
	\begin{itemize}[noitemsep]
		\item $\mathcal{M}$ is the type map, a dictionary that maps type
			variables to explicit types (can also be denoted
			$\mathcal{M}(C_f)$);
		\item $\mathcal{A}$ is the atlas, a dictionary that maps paths in the
			call graph to function implementations;
		\item $\mathcal{W}$ is the type variable map, a dictionary that maps
			type variables to type variables;
		\item $\mathcal{G}$ is the set of child functions;
		\item $\mathcal{X}$ is a dictionary that maps variable name to type
			variable;
		\item $\mathcal{L}$ is a set of references to the contexts of parent
			functions of $f$;
		\item $\mathcal{K}$ maps type variables to a subset of type classes
			(the type classes that the type variable may be). \texttt{Any} will
			be used if there is no restriction, it will act as the universal
			set of type classes. \texttt{Resolved} will be used if the type was
			explicit.
		\item $\mathcal{E}$ is a set of errors that $f$ throws or that were
			thrown by child functions; and
		\item $\mathcal{N} \in \mathbb{Z}^+$ is the number of parameters
			including the return type.
	\end{itemize}

	For brevity, we also use the notations $\mathcal{Y}(C_f)$ and
	$\mathcal{Y}_f$ for some tuple-member $\mathcal{Y}$ and function $f$.

	We may write $V_0 = f(V_1, V_2, \cdots, V_{\mathcal{N}-1})$ for $V_i \in
	\mathcal{T}$ (set of types). Exact definitions for the elements of the
	tuple will follow below.
	
	\paragraph{Type}
	A type is a named set of values. Familiar types in programming languages
	would include \texttt{int} or \texttt{float}. We will
	use lowercase roman characters near the beginning of the alphabet to denote
	types and resort to subscripts if need be. The set of types will be denoted
	$\mathcal{T}$.
	
	\paragraph{Type Variable}
	A type variable is used in lieu of an explicit type to denote parametric
	polymorphism. We will use uppercase $T$, $R$, and $S$ to denote type
	variables and use subscripts when necessary. The set of type variables will
	be denoted $\mathcal{V}$.
	
	\paragraph{Type Class}
	A type class is a set of types (a subset of the set of types). We denote
	type classes with uppercase roman characters near the beginning of the
	alphabet. If $a \in A$, we say that $a$ implements $A$. The set of type
	classes is denoted $\mathcal{C}$.
	
	\paragraph{Up-type}
	An up-type is equivalent to an ``error type.''  Up-types explicitly travel
	from callees to callers. Error types will only be denoted in code in the
	form ``ErrorName'' preceded by the keywords ``throws''.

	Up-types are a subset of the set of types. They cannot be used as return
	types or parameter types.
	
	\paragraph{Down-type}
	A down-type is a type which is passed from caller to callee. Theis
	corresponds with the conventional definition of types. 
	
	\paragraph{Incomplete type}
	An incomplete type is any type which is represented by a type variable.
	Incomplete types are those types which we need to be completed by Paratype.
	
	
\subsection{Graph Terminology}
	\paragraph{Call Graph}
	The call graph represents which functions call each other. It conveys
	information about the relationship between functions, but may not describe
	the relationship between type variables of functions. Paratype is actually
	analyzing the \emph{type variable graph}. However, the call graph is useful
	for simple examples and may be used in discussion.
	
	\paragraph{Parent}
	A parent function is a function which is calling a child function (parent
	node in call graph). If $f$ is the parent of $g$, we will use the notation
	$f \circ g$.
	
	\paragraph{Child}
	A child function is a function which is called by a parent (child node in
	call graph). If $f$ is the child of $g$, we will use the notation $g \circ
	f$.
	
	\paragraph{Function Composition}
	Function composition is the pattern of using a function call as an argument
	to another function. For single argument examples the common $f \circ
	g$ notation will be used.
	
	\paragraph{Context Tree}
	The tree representing every path in the call graph. A node of the context
	tree represents one path in the call graph. Because the set of types is a
	finite set, any cycles in the call graph will eventually unravel and the
	context graph will be a context tree.
	
	\paragraph{Typevar Graph}
	The type variable (typevar) graph is the graph which Paratype attempts to
	solve. It describes the relationship between all type variables in the
	source file.
	
\subsection{Algorithm Terminology}
	\paragraph{Type Map $\mathcal{M}(C_f)$}
	A dictionary with type variables as keys and explicit types as values. For
	those type variables which do not yet have explicit type we will use
	$\epsilon$ (\texttt{incomplete}).
	
	\[
	\mathcal{M}(C_f) \subseteq \{ (v, t) : t \in \mathcal{T}
	\cup \{\epsilon\}, v \in \mathcal{V}\textrm{ and $v$ part of some path in
	}\mathcal{A}(C_f) \}.
	\]
	 
	\paragraph{Atlas $\mathcal{A}(C_f)$}
	A dictionary which maps a path in the call tree that starts with $f$ to a
	tuple of type variables.
	
	Let $(p, V)$ be the ordered pair where
	\begin{itemize}[noitemsep]
		\item $p$ is a path in the call tree. Let the end of that path be the
			function $g$.
		\item $V = (V_0, V_1, \cdots, V_{n-1})$ where $V_0 = g(V_{1}, V_{2},
			\ldots, V_{n-1})$ and $V_{i} \in \mathcal{M}(C_g)$ for all $i$.
	\end{itemize}
	\[
		\mathcal{A}(C_f) \subseteq \{ (p , V) : p\textrm{ begins with }f\}.
	\]
	The keys of the dictionary (paths) are unique.

%	A dictionary which maps paths in the call tree to a list of a list of type
%	variables. Each element of the outer-most list  represents a particular
%	function call in the context path. The $n$th element of the inner-most list
%	represents the $n$th argument's type variable.  
	In the algorithm, the atlas is used to
	identify identical type variables and to perform type checking.

	\paragraph{Type Variable Map $\mathcal{W}(C_f)$}
	A dictionary that maps type variables to type variables. To be used by $f$
	when it has replaced a type variable and needs to keep replacing that same
	type variable again in other functions.

	\paragraph{Parent Functions $\mathcal{L}(C_f)$}
	List of references to contexts of all callees of $f$.

	\paragraph{Child Functions $\mathcal{G}(C_f)$}
	A set of child function names. Composed function calls are considered child
	functions.

	\paragraph{Variable Dictionary $\mathcal{X}(C_f)$}
	Dictionary that maps variable names to type variables.

	\paragraph{Errors $\mathcal{E}(C_f)$}
	A set of errors that either $f$ throws or was thrown by a child function of
	$f$.

	\paragraph{Context Update}
	A context update is the stage in our algorithm in which the type
	information is updated in a context, and type variables may be merged in
	the type map.

	\paragraph{In-Out Evaluation} 
	In-Out Evaluation describes the necessity to
	evaluate the context of inner-most function calls before outer function
	calls in function composition. For example, $f(g(x),h(x))$ must evaluate
	$g$ and $h$ before it can convey context information to $f$.

\pagebreak

\section{Grammar}

\subsection{GOPP}

Following is the grammar of our language specified in the BNF-like
\texttt{gopp}\footnote{See \url{https://github.com/skelterjohn/gopp}} format.

\lstinputlisting[language=Gopp]{paratype.gopp}

%\subsection{EBNF}
%
%Following is the grammar of our language in EBNF.
%
%\begin{grammar}
%    <lower-letter> ::= `...';
%
%    <upper-letter> ::= `...';
%
%    <type-name> ::= <lower-letter>, \{<upper-letter> |
%     <lower-letter>\};
%
%    <error-name> ::= <lower-letter>, \{<upper-letter>
%     | <lower-letter>\};
%
%    <typeclass-name> ::= <upper-letter>, \{<upper-letter>
%     | <lower-letter>\};
%
%    <type-var> ::= <upper-letter>, \{<upper-letter>\};
%
%    <var> ::= <lower-letter>, \{<lower-letter>\}
%
%    <func-name> ::= <lower-letter>, \{<upper-letter>
%     | <lower-letter>\};
%
%    <type-place> ::= <type-var> | <type-name>
%
%    <func-arg> ::= <type-place>, ` ', <var>
%
%    <func-args> ::= `(', \{<func-arg>, `, '\}, <func-arg>, `)';
%
%    <func-constraint> ::= <type-var>, ` to ', <typeclass-name>,
%     \{` ', <typeclass-name>\};
%
%    <func-constraints> ::= `constrain ', <func-constraint>,
%     \{`, ', <func-constraint>\};
%
%    <func-errors> ::= ` throws ', <error-name>,
%     \{`, ', <error-name>\};
%
%    <typeclass-inherit> ::= ` inherits ', <typeclass-name>,
%     \{`, ', <typeclass-name>\};
%
%    <type-implement> ::= ` implements ', <typeclass-name>,
%    \{`,', <typeclass-name>\};
%
%    <type-decl> ::= `type ', <type-name>, [<type-implement>];
%
%    <typeclass-decl> ::= `typeclass ',
%    <typeclass-name>, [<typeclass-inherit>];
%
%    <expr> ::= <type-name>
%    \alt <type-var>
%    \alt <func-name>, `(', [\{<expr>, `,'\}, <expr>], `)';
%
%    <func-sig> ::= `func ', <func-name>, [` ', <func-constraints>], ` ',
%    <func-args>, ` ', <type-place>, [` ', <func-errors>];
%
%    <func-dec> ::= <func-sig>, `\textbackslash n =', <expr> 
%\end{grammar}

\end{document}
