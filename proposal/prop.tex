% THIS IS SIGPROC-SP.TEX - VERSION 3.1
% WORKS WITH V3.2SP OF ACM_PROC_ARTICLE-SP.CLS
% APRIL 2009
%
% It is an example file showing how to use the 'acm_proc_article-sp.cls' V3.2SP
% LaTeX2e document class file for Conference Proceedings submissions.
% ----------------------------------------------------------------------------------------------------------------
% This .tex file (and associated .cls V3.2SP) *DOES NOT* produce:
%       1) The Permission Statement
%       2) The Conference (location) Info information
%       3) The Copyright Line with ACM data
%       4) Page numbering
% ---------------------------------------------------------------------------------------------------------------

\documentclass{acm_proc_article-sp}
\usepackage{listings,color,xcolor}
\definecolor{dkgreen}{rgb}{0,0.6,0}

\lstdefinestyle{paratype}{
	breaklines=true,
	numbers=left,
	basicstyle=\small\ttfamily,
	keywords={func,inherits,implements,throws,type,typeclass},
	keywordstyle=\bfseries\color{dkgreen}, 
	commentstyle=\itshape\color{purple},
	identifierstyle=\color{black}, 
	stringstyle=\color{blue}, 
	numberstyle=\scriptsize, 
	frame=single,
	frameround=tttt,
	tabsize=4,
	showstringspaces=false,
	aboveskip=1.8em, 
	belowskip=0em,
	captionpos=b,
	xleftmargin=1.8em,
	xrightmargin=0.5em
}

\begin{document}

\title{Paratype --- A Parallel Type Completion System}


% You need the command \numberofauthors to handle the 'placement
% and alignment' of the authors beneath the title.
\numberofauthors{3} 
\author{
\alignauthor
Tyler Cecil\\
       \affaddr{New Mexico Tech}\\
       \affaddr{801 Leroy Place}\\
       \affaddr{Socorro, NM 87801 USA}\\
       \email{tcecil@nmt.edu}
\alignauthor
Ben Turrubiates\\
       \affaddr{New Mexico Tech}\\
       \affaddr{801 Leroy Place}\\
       \affaddr{Socorro, NM 87801 USA}\\
       \email{bturrubi@nmt.edu}
\alignauthor
Christopher Koch\\
       \affaddr{New Mexico Tech}\\
       \affaddr{801 Leroy Place}\\
       \affaddr{Socorro, NM 87801 USA}\\
       \email{ckoch@cs.nmt.edu}
}
\date{\today}

\maketitle
\begin{abstract}
\end{abstract}

% A category with the (minimum) three required fields
\category{H.4}{Information Systems Applications}{Miscellaneous}
%A category including the fourth, optional field follows...
\category{D.2.8}{Software Engineering}{Metrics}[complexity measures, performance measures]

\terms{Theory}

\keywords{type theory, language theory, type completion, type checking, type
inference, code analysis}

\section{Introduction}

% problem and why important
% compilers: e.g. just-in-time comp

\section{Problem Definition}

% grammar, input/output, type completion, lambda calculus (System F)

Formally, our problem will be to take an input file of a specified grammar and
generate an output of either failure due to indecidability or the same file
with explicit types instead of type variables. In the process of doing this, we
will also be performing typechecking. 

We will define a few terms to use throughout the proposal:

\begin{description}
	\item[Type] A set of values. Types may implement type classes.
	\item[Type class] A set of types. Typeclasses may inherit other type classes.
	\item[Context] The set of explicit types and metainformation associated
		with a function call: caller, parameter types, return type,
		and a list of functions who care about the context.
	\item[Partial context] A context where not all types are known explicitly.
\end{description}

\begin{lstlisting}[caption=Grammar displayed informally,style=paratype]
typeclass T inherits T1, T2, ...

type A implements T
    = Type | (Type, Type, ...)

func foo(Type var, Typevar var2, ...)
    Type throws Error1, Error2, Error3
    = bar(baz(var), var2)
\end{lstlisting}

\section{Parallelization}

% actor model, functions as nodes, DAG

\section{Methods}

% Google Go and why

\section{Analysis and Model}

% variables: number of functions, function calls, contexts, partial contexts,
% indecidability, number of physical cores available, number of types and
% typeclasses, etc

\section{Team and Timeline}

\section{Conclusion}

% why appropriate project?

%\balancecolumns
\end{document}
