% THIS IS SIGPROC-SP.TEX - VERSION 3.1
% WORKS WITH V3.2SP OF ACM_PROC_ARTICLE-SP.CLS
% APRIL 2009
%
% It is an example file showing how to use the 'acm_proc_article-sp.cls' V3.2SP
% LaTeX2e document class file for Conference Proceedings submissions.
% ----------------------------------------------------------------------------------------------------------------
% This .tex file (and associated .cls V3.2SP) *DOES NOT* produce:
%       1) The Permission Statement
%       2) The Conference (location) Info information
%       3) The Copyright Line with ACM data
%       4) Page numbering
% ---------------------------------------------------------------------------------------------------------------

\documentclass{acm_proc_article-sp}
\usepackage{listings,color,xcolor}
\usepackage{pgf,tikz,pgfgantt}
\usepackage{lmodern} % font problem fix

\usepackage{subcaption}
\usepackage{graphicx}
\usepackage{float,dblfloatfix,fixltx2e}
\usepackage[format=plain,font=small,labelfont=bf]{caption}
\usepackage[utf8]{inputenc}
\usepackage{syntax, etoolbox}
\usepackage{lstlang0}

\definecolor{dkgreen}{rgb}{0,0.6,0}
\AtBeginEnvironment{grammar}{\small}

\usepackage{MnSymbol}
\lstset{
	prebreak=\raisebox{0ex}[0ex][0ex]{\ensuremath{\rhookswarrow}},
	postbreak=\raisebox{0ex}[0ex][0ex]{\ensuremath{\rcurvearrowse\space}},
	breaklines=true,
	numbers=left,
	numberstyle=\scriptsize,
	breakatwhitespace=true,
	frame=single,
	frameround=tttt,
	tabsize=4,
	showstringspaces=false,
	aboveskip=1.8em,
	belowskip=0em,
	captionpos=b,
	xleftmargin=0.4em,
	xrightmargin=0em,
	keywordstyle=\bfseries\color{dkgreen},
	commentstyle=\itshape\color{purple},
	identifierstyle=\color{black},
	stringstyle=\color{blue},
	basicstyle=\small\ttfamily
}

\lstdefinelanguage{Paratype}{
	morekeywords={func,inherits,implements,throws,type,typeclass,contrain,to},
}

\newganttchartelement{chris}{
	chris/.style={
		color=black,
		fill=red,
		shape=ganttbar,
		inner sep=0pt,
		draw
	},
	chris height=0.5
}
\newganttchartelement{tyler}{
	tyler/.style={
		color=black,
		fill=green,
		shape=ganttbar,
		inner sep=0pt,
		draw
	},
	tyler height=.5
}
\newganttchartelement{ben}{
	ben/.style={
		color=black,
		fill=blue,
		shape=ganttbar,
		inner sep=0pt,
		draw
	},
	ben height=.5
}

\setlength{\grammarparsep}{8pt plus 1pt minus 1pt}
\setlength{\grammarindent}{12em}

\begin{document}

\title{Paratype --- A Parallel Type Completion System}


% You need the command \numberofauthors to handle the 'placement
% and alignment' of the authors beneath the title.
\numberofauthors{3}
\author{
\alignauthor
Tyler Cecil\\
       \affaddr{New Mexico Tech}\\
       \affaddr{801 Leroy Place}\\
       \affaddr{Socorro, NM 87801 USA}\\
       \email{tcecil@nmt.edu}
\alignauthor
Ben Turrubiates\\
       \affaddr{New Mexico Tech}\\
       \affaddr{801 Leroy Place}\\
       \affaddr{Socorro, NM 87801 USA}\\
       \email{bturrubi@nmt.edu}
\alignauthor
Christopher Koch\\
       \affaddr{New Mexico Tech}\\
       \affaddr{801 Leroy Place}\\
       \affaddr{Socorro, NM 87801 USA}\\
       \email{ckoch@cs.nmt.edu}
}
\date{\today}

\maketitle
\begin{abstract}
  In the following proposal we introduce \emph{Paratype}, an actor model of
  type completion. Type analysis of modern languages such as
  Haskell, Agda, and Coq can be an intense stage in compilation. With the use
  of an actor model, we hope to cut down compilation time as well as open up
  doors for new models of type evaluation. Working on a grammar with bounded
  parametric polymorphism, Paratype will communicate types across a large call
  graph to resolve type errors, append errors, and evaluate types in parallel.
  To minimize message overhead, we propose to use a shared memory message
  passing system. We believe our actor approach can significantly
  speed up the compilation process as well as apply to some other interesting
  problems.
\end{abstract}

\category{F.3.3}{Studies of Program Constructs}{Type structure}
\category{D.3.3}{Programming Languages}{Language Constructs and Features}
\category{D.3.4}{Programming Languages}{Processors}

% some out of 16 general terms!
\terms{Design, Languages, Performance}

\keywords{type theory, language theory, type completion, type checking, type
inference, code analysis, parallel}

\section{Introduction}

% problem and why important
% compilers: e.g. just-in-time comp

``Compile one time, run many times: that means compilers do not need to be
fast, right?'' Luckily, no software engineer thinks this way. In fact,
compilation speed is a concern of many budding languages. Aside from lexing
(typically the slowest part of compilation)~\cite{bright:blog}, modern
languages spend an embarrassing amount of time resolving
types~\cite{junaidu:phd}. Principally, this is due to the adoption of the
\emph{System F} formalization in modern strongly typed languages. Moreover,
many languages have type inference or even type bounds, further complicating
the process. Unfortunately, type evaluation serves as the foundation for
automated theorem provers and other logic analysis systems. Even within the
compilation domain, interpreted languages and JIT languages need type
resolution much faster than is currently being provided. Type evaluation needs
to be expedited.

To this end, we propose the development of \emph{Paratype}: an actor model of
type evaluation. As will be apparent shortly, most type evaluation procedures
can be described as a conversation between functions. Each function
communicates information about its type to ``neighboring'' functions. As the
chatter subsides, we are left with all functions knowing their types or a
function knowing that it cannot compile. Most systems involve walking through
the tree of this conversation. We propose to let this natural model of
communication play itself out. Using a shared memory message passing system,
\emph{Paratype} will allow functions to be actors and let the conversation
happen.

Initially we are only going to use a toy grammar that is defined in this
proposal. Our grammer will include down types (normal variables), and up types
(error types) in order to cover most use cases in modern type systems. Perhaps
one day \emph{Paratype} can be integrated into a project such as the Haskell
compiler \texttt{ghc}.


\section{Problem Definition}
\label{sec:problem}
% grammar, input/output, type completion, lambda calculus (System F)

Formally, our problem will be to take an input file of a specified grammar and
generate an output of either failure due to undecidability or the same file
with explicit types instead of type variables. In the process of doing this, we
will also be performing type checking. The goal is to find an explicit type for
each function call without type conflicts. Please see Appendix~\ref{app:Grammar}
for our EBNF grammar.

We define a few terms to use throughout the proposal:
\begin{description}
	\item[Type] A set of values. Types may implement type classes.
	\item[Type class] A set of types. Type classes may inherit other type classes.
	\item[Context] The set of explicit types and metainformation associated
		with a function call: caller, parameter types, return type,
		and a list of functions who care about the context.
	\item[Partial context] A context where not all types are known explicitly.
	\item[Resolution] Resolving a context means giving it explicit types.
	\item[Parent (function)] Caller of a function.
	\item[Child (function)] A function that is called by its parent function.
	\item[Type variable] A variable that ranges over types.
	\item[Parametric polymorphism] A way to allow a language to express the
		handling of functions and types homogeneously independent of their type
		through the use of generic functions, also known as generic
		programming.
	\item[Bounded parametric polymorphism] A method of providing extra
		information about a generic data type. In Haskell, this is achieved by
		creating a type class which types may implement.
\end{description}

Listing~\ref{lst:informalg} shows an informal example of our grammar. It is a
grammar for a simple functional language that provides bounded parametric
polymorphism through type classes. This allows expressing functions in terms of
generic types while still maintaining the same behavior. This use of type
variables introduces the need for partial contexts.

\begin{lstlisting}[caption=Grammar displayed informally,language=Paratype,label=lst:informalg]
typeclass num inherits tc1, tc2, ...

type int implements num
    = othertype | (otype, otype, ...)

func foo(otype var, Typevar var2, ...)
    otype throws error1, error2, error3
    = bar(baz(var), var2) | otype
\end{lstlisting}

An example of a partial context is shown in Listing~\ref{lst:cbyparent}. The
function \lstinline!bar! is declared as accepting and returns a \lstinline!T!
type.  This introduces a partial context due to \lstinline!T! being a type
variable. The parameter type for \lstinline!bar! is provided by its parent
function \lstinline!foo!. This also completes the return type of
\lstinline!bar!.

\begin{lstlisting}[caption=Explicit context provided by parents,language=Paratype,label=lst:cbyparent]
func foo(int x, int y) int
    = bar(y)

func bar(T d) T
    = T
\end{lstlisting}

Types for partial contexts can be provided by both the parent and the child.
Consider the example shown in Listing~\ref{lst:cbychild}. Function
\lstinline!foo!  is defined as returning an \lstinline!R! type. The function
definition contains the result of calling \lstinline!bar! as the return value.
\lstinline!bar! is defined as having an \lstinline!int! return type. Since
\lstinline!foo! returns the value of calling \lstinline!bar! it also has a
return type of \lstinline!int!.

\begin{lstlisting}[caption=Explicit context provided by child,language=Paratype,label=lst:cbychild]
func foo(T a, S b) R
    = bar(a, b)

func bar(float a, float b) int
    = int
\end{lstlisting}

In the previous examples all partial contexts have been resolved by either the
parent or the child. There are situations where they can mutually provide
contexts for each other. One example of this is shown in
Listing~\ref{lst:cbyboth}. In this example \lstinline!foo! is providing
the type of the second parameter for function \lstinline!barbar!.
\lstinline!barbar!  also provides the type of the first parameter for function
\lstinline!foo!. Notice that function \lstinline!barbar! is defined as
returning a type variable \lstinline!R!. Since \lstinline!foo! returns an
\lstinline!int! type and its return value is a call to \lstinline!barbar!; this
resolves the return type for \lstinline!barbar!. The evaluation of these
partial contexts is a non-trivial task.

\begin{lstlisting}[caption=Explicit context provided by child and parent,language=Paratype,label=lst:cbyboth]
func foo(T a, int b) int
    = barbar(a, b)

func barbar(int a, T b) R
    = R
\end{lstlisting}

Another problem that type variables introduce is the need to maintain partial
contexts. Multiple fully evaluated contexts may exist in the end and may not be
the result of type errors. An example of this is shown in
Listing~\ref{lst:partial}. The \lstinline!bar! function accepts a type variable
\lstinline!b! as an input parameter. The function \lstinline!foo! provides a
context for \lstinline!bar! in which \lstinline!b! is of type \lstinline!int!.
At this point \lstinline!bar! now has a fully evaluated context, but
\lstinline!baz! can also complete the partial context with \lstinline!b!
resolving to type \lstinline!float!.  In this situation these are both valid
and should not be considered a type error.

\begin{lstlisting}[caption=Need to maintain partial contexts,language=Paratype,label=lst:partial]
func foo(int x) T
    = bar(x)

func bar(T b) char
    = char

func baz(float b) T
    = bar(b)
\end{lstlisting}

Although contexts can be provided by both the parent and the child, there are
situations where there is not enough information to resolve a context. Consider
Listing~\ref{lst:unhalting}: \lstinline!bar! accepts a type variable as a
parameter. The context provided by its calling function \lstinline!foo!
resolves the parameter as being the local variable \lstinline!b!. This is
problematic due to \lstinline!b! also being a type variable. In this example
there is not enough information to resolve a full context for either
\lstinline!bar! or \lstinline!foo!.

\begin{lstlisting}[caption=Unhalting context resolution,language=Paratype,label=lst:unhalting]
func foo(T b) int
    = bar(b)

func bar(T a) int
    = int
\end{lstlisting}

Introducing errors into the grammar adds more complexity since they need to
propagate up the call stack. In the example shown in Listing~\ref{lst:errors},
all parent functions of \lstinline!bar! need to have \lstinline!weirdError! in
their fully resolved contexts.

\begin{lstlisting}[caption=Errors,language=Paratype,label=lst:errors]
func foobar(T b) T
    = foo(b)

func foo(int b) int
    = bar(b)

func bar(T a) T throws weirdError
    = T
\end{lstlisting}

% ADD EXAMPLE OF NESTED FUNCTIONS

\section{Parallelization}

Paratype will act as an \emph{actor model} of computation for our type
system: \emph{every function will be an actor}. We feel that not only will
such a model be efficient for type resolution, but it also acts as a
natural solution to the problem. The type resolution process is
essentially a conversation between functions: a function asks others for
information about their type in order to resolve its own type. We take
advantage of this natural communication and process the problem in
parallel.

Unfortunately, the communication is not linear as shown by our
previous examples. This makes our task quite a bit more challenging,
especially if we want to keep communication overhead low! Each node
needs to communicate to its parents and children, send partial
contexts, and update error types. Actors also need to know when their
process should finish, compile, or error out. This complexity,
however, is not all for not.

\subsection{Expected Speedup}

Where does speedup come from, other than the parallel computation of
contexts? Consider a chain of function calls in which the last
function in the chain has a type constraint. In the serial case, types
would need to propagate all the way down to check for a type error. In
the parallel case, however, contexts move both up and down
concurrently. Meeting halfway, we can check for type errors. This
is only one factor and it only matters in invalid code. However, the
chains themselves can now be evaluated in parallel. If we are even
more clever, we can use shared memory for our contexts and thus allow
changes that occur at one end of a call graph to instantly propagate
to all other nodes without unnecessary message overhead. When
considering the factors that provide speedup for our application, it
becomes apparent that our speedup is a function of many attributes:
call graph width, number of functions, invalid versus valid code, etc.
These need to be considered when building our model.

In some ways, speedup will be hard to measure: we do not have a serial
implementation for our toy language. Due to time constraints, we
have omitted implicit typing from our grammar (not that our system
could not handle it), which prevents us from simply resolving Haskell
types and comparing our results to \texttt{ghc} and
\texttt{hugs}. This is a future goal for us. At present, however,
we will be able to vary the number of processing nodes being
utilized via \texttt{GOMAXPROCS}.

\subsection{Expected Problems}

Moving contexts around is no trivial task. A huge amount of
bookkeeping will be required to prevent deadlock, hanging, unnecessary
contexts, and a host of other possible problems. Marking down where a
context came from is not even trivial: through composition of
function calls, an inner function of that composition may receive
contexts through wrapping functions of that composition. While none of
these problems are insurmountable, they require our code to be
constructed carefully.

%TODO: USE AN ACTUALL LABEL
Our examples in Section~\ref{sec:problem} show a number of problems that must be
resolved in our method. Perhaps the most daunting is knowing when
execution is complete. More subtle, however, are problems that
come from having no ``main'' function. In a traditional compiler
setting, we have a function to begin propagation from. We, however,
evaluate all contexts in parallel. With this lack of a starting point,
some communications become ambiguous: do we communicate down or up
the call tree? Do we wait to be signaled? There are many more questions. These
are all problems that will need to be resolved by \emph{Paratype}.

\section{Methods}

% Google Go and why

Due to the actor model, the problem lends itself nicely to a distributed memory
approach; however, it is usually commodity hardware that is used for
compilation. Therefore, we propose to use Google Go due to its concept of
\texttt{goroutines}. A \texttt{goroutine} is a lightweight thread that is
managed by Go during runtime.

\begin{lstlisting}[caption=Small \texttt{goroutine} example with
\texttt{channels},language=Go,label=lst:goroutine]
func node(name string, c chan string) {
	fmt.Println("I'm node ", name)
	/* send message to c
	 * (blocks until receiver is ready)
	 */
	c <- strings.Join(name, " has a message for you.")
}

func main() {
	// unbuffered channel
	c := make(chan string)
	go node("abc", c)
	// receive message from c (blocks)
	a := <-c
	fmt.Println(a)
}
\end{lstlisting}

It is possible to pass references through shared memory, called
\texttt{channels}, with \texttt{goroutines}: like message passing in shared
memory. This can be seen in Listing~\ref{lst:goroutine}. Of course, unlike in
the example, the channel may also be buffered and would thus only block sending
when the buffer is full and receiving when the buffer is empty. This is one of
the reasons that we chose Google Go: it allows us to use message passing
without the overhead of actually sending the message.

The \texttt{goroutines} fit the actor model of type resolution nicely since
the threads are lightweight and communication is easy to arrange. We may even
take advantage of the fact that not messages but references are passed through
\texttt{channels}. This way, a specific context will update everywhere that it
is referenced and we must only notify other actors that something changed
instead of sending the context again. This reduces communication overhead while
still functioning as a message passing system.

\section{Analysis and Model}

We will show the correctness of our program by having an input generator that
generates both a source file and a compiled file with explicit types. We can
then compare the generated compiled file to the file that our program
generates. 

Our program can be modeled with many variables: number of functions, number of
function calls, number of contexts, number of total partial contexts in
existence during runtime, the number of physical cores available, the number of
types and typeclasses, call graph width, and undecidability among many other
things.

The model needs to make a distinction between runtime for valid and invalid
code: when a type completion becomes undecidable, the program will likely
behave much differently than when it is resolvable. It is also likely that some
parameters will unexpectedly influence the model: while the number of functions
determines the number of threads, it is the number of function calls that
determines how many communications will happen. Yet, if we limit the number of
processing nodes via \texttt{GOMAXPROCS}, the number of functions will take
greater influence on running time.

% variables: number of functions, function calls, contexts, partial contexts,
% undecidability, number of physical cores available, number of types and
% type classes, etc

\begin{figure*}
	% spans two columns
\centering
\begin{ganttchart}[
    vgrid,
	time slot format/start date=2014-04-14,
	time slot format=isodate,
	bar height=.5,
	y unit chart=0.6cm,
    %today=9,
    %bar/.append style={fill=green},
	%bar tyler/.append style={fill=green},
	%bar ben/.append style={fill=blue},
	%bar chris/.append style={fill=red}
]{2014-04-09}{2014-05-02}
\gantttitlecalendar{month=name, day, week=15} \\
\ganttbar{Proposal due}{2014-04-09}{2014-04-09}\\
\ganttbar{Design objects and communications}{2014-04-09}{2014-04-14}\\
\ganttchris{Error types}{2014-04-09}{2014-04-15}\\
\ganttben{}{2014-04-14}{2014-04-15}\\
\gantttyler{A Tyler task}{2014-04-14}{2014-04-15}\\
\ganttbar{A group task}{2014-04-14}{2014-04-15}
\end{ganttchart}
\vspace{3mm} % fix weird caption placement
\caption{Timeline: tasks for Tyler (green), Ben (blue), Chris (red), and the
group (white)}
\label{fig:schedule}
\end{figure*}

\section{Team and Timeline}

Our group consists of three team members: Tyler, Ben, and Chris.
Figure~\ref{fig:schedule} depicts a Gantt chart that shows all team members'
tasks.

We have developed a list of goals ordered by priority:
\begin{description}
	\item[First order goals]
          \hfill
          \begin{itemize}
            \item Construct working Function Actors 
            \item Function Actors can communicate Contexts to eachother
            \item Invalid Type detection
            \item Error Types communicated
            \item Program deterministically halts
          \end{itemize}

	\item[Second order goals]
          \hfill
          \begin{itemize}
            \item Resolve partial contexts
            \item Parse Input files
            \item Check Output files
            \item Remove unnessesary communication
          \end{itemize}

	\item[Third order goals]
          \hfill
          \begin{itemize}
            \item Type Inferrence
            \item Haskell Compliance
            \item GHC integration
          \end{itemize}
\end{description}

\section{Conclusion}

\appendix
\section{EBNF Grammar Definition}
    \label{app:Grammar}
    \begin{grammar}

    <lower-letter> ::= `...';

    <upper-letter> ::= `...';

    <type-name> ::= <lower-letter>, \{<upper-letter> |
    \\ <lower-letter>\};

    <error-name> ::= <lower-letter>, \{<upper-letter>
    \\ | <lower-letter>\};

    <typeclass-name> ::= <upper-letter>, \{<upper-letter>
    \\ | <lower-letter>\};

    <type-var> ::= <upper-letter>, \{<upper-letter>\};

    <var> ::= <lower-letter>, \{<lower-letter>\}

    <func-name> ::= <lower-letter>, \{<upper-letter>
    \\ | <lower-letter>\};

    <type-place> ::= <type-var>
    \alt <type-name>

    <func-arg> ::= <type-place>, ` ', <var>

    <func-args> ::= `(', \{<func-arg>, `, '\}, <func-arg>, `)';

    <func-constraint> ::= <type-var>, ` to ', <typeclass-name>,
    \\ \{` ', <typeclass-name>\};

    <func-constraints> ::= `constrain ', <func-constraint>,
    \\ \{<func-constraint>\};

    <func-errors> ::= ` throws ', <error-name>,
    \\ \{`, ', <error-name>\};

    <typeclass-inherit> ::= ` inherits ', <typeclass-name>,
    \\ \{`, ', <typeclass-name>\};

    <type-implement> ::= ` implements ', <typeclass-name>,
     \\\{`,', <typeclass-name>\};

    <type-decl> ::= `type ', <type-name>, [<type-implement>];

    <typeclass-decl> ::= `typeclass ',
    \\<typeclass-name>, [<typeclass-inherit>];

    <expr> ::= <type-name>
    \alt <type-var>
    \alt <func-name>, `(', [\{<expr>, `,'\}, <expr>], `)'

    <func-sig> ::= `func ', <func-name>, [<func-constraints>], 
    <func-args>, <type-place>, [<func-errors>];

    <func-dec> ::= <func-sig>, `\textbackslash n =', <expr> 

    \end{grammar}


% why appropriate project?

\balancecolumns
\end{document}
