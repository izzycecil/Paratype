% THIS IS SIGPROC-SP.TEX - VERSION 3.1
% WORKS WITH V3.2SP OF ACM_PROC_ARTICLE-SP.CLS
% APRIL 2009
%
% It is an example file showing how to use the 'acm_proc_article-sp.cls' V3.2SP
% LaTeX2e document class file for Conference Proceedings submissions.
% ----------------------------------------------------------------------------------------------------------------
% This .tex file (and associated .cls V3.2SP) *DOES NOT* produce:
%       1) The Permission Statement
%       2) The Conference (location) Info information
%       3) The Copyright Line with ACM data
%       4) Page numbering
% ---------------------------------------------------------------------------------------------------------------

\documentclass{acm_proc_article-sp}

\begin{document}

\title{Paratype}
\subtitle{A Parallel Type Completion System}

% You need the command \numberofauthors to handle the 'placement
% and alignment' of the authors beneath the title.
\numberofauthors{3} 
\author{
\alignauthor
Tyler Cecil\\
       \affaddr{New Mexico Tech}\\
       \affaddr{801 Leroy Place}\\
       \affaddr{Socorro, NM 87801 USA}\\
       \email{tcecil@nmt.edu}
\alignauthor
Ben Turrubiates\\
       \affaddr{New Mexico Tech}\\
       \affaddr{801 Leroy Place}\\
       \affaddr{Socorro, NM 87801 USA}\\
       \email{bturrubi@nmt.edu}
\alignauthor
Christopher Koch\\
       \affaddr{New Mexico Tech}\\
       \affaddr{801 Leroy Place}\\
       \affaddr{Socorro, NM 87801 USA}\\
       \email{ckoch@cs.nmt.edu}
}
\date{\today}

\maketitle
\begin{abstract}
\end{abstract}

% A category with the (minimum) three required fields
\category{H.4}{Information Systems Applications}{Miscellaneous}
%A category including the fourth, optional field follows...
\category{D.2.8}{Software Engineering}{Metrics}[complexity measures, performance measures]

\terms{Theory}

%\keywords{type theory, language theory, type completion, type checking, type
%inference, code analysis} % NOT required for Proceedings

\section{Introduction}

% problem and why important

\section{Problem Definition}

% grammar, input/output, type completion, lambda calculus (System F)

\section{Parallelization}

% actor model, functions as nodes, DAG

\section{Methods}

% Google Go and why

\section{Analysis and Model}

% variables: number of functions, function calls, contexts, partial contexts,
% indecidability, number of physical cores available, number of types and
% typeclasses, etc

\section{Team and Timeline}

\section{Conclusion}

% why appropriate project?

%\balancecolumns
\end{document}
